\chapter{Ergebnis, Fazit und Ausblick}
\section{Ergebnis}
Im Kontrollsystem einer EEROS Applikation können Blöcke implementiert werden, die Informationen von beliebigen ROS \textit{Topics} lesen können.
Wenn ein externer Sensor, zum Beispiel ein Laserscanner, auf ein \textit{Topic} seine Messdaten veröffentlicht, kann eine EEROS Applikation diese Daten im Kontrollsystem nutzen.
Es können auch Daten aus dem Kontrollsystem auf einem beliebigen ROS \textit{Topic} veröffentlicht werden, um sie dann mit \textit{rviz} oder einem anderen ROS Tool zu visualisieren oder abzuspeichern.

In EEROS wurde auch ein neuer Block implementiert, der einfache Signale inklusive Zeitstempel aus dem Kontrollsystem kontinuierlich und verlustfrei auf einem \textit{Topic} veröffentlicht.
Die veröffentlichten Daten können dann mit bestehenden ROS Programmen wie \textit{rqt-multiplot} visualisiert und abgespeichert werden.
Dies ist besonders für die Fehlersuche ein nützliche Funktion.

Die EEROS HAL kann auch verwendet werden um Daten von einem \textit{Topic} zu lesen, oder um Daten auf einem \textit{Topic} zu veröffentlichen.

Neu ist es auch möglich, eine EEROS-Applikation mit einer \textit{Gazebo} Simulation zu testen.
Selbst wenn die Simulation nicht in Echtzeit läuft, kann die EEROS Applikation mit der Simulation synchronisiert werden.
Auch die Zeitstempel werden richtig berechnet.


\section{Fazit}	%subjektiv
Bei meiner letzten Vertiefungsarbeit hatte ich versucht, alle Änderungen für die Software auf einen Schlag zu implementieren.
Ich hatte mein ganzes Konzept als Pseudocode durchgedacht und dann versucht, alles in der Software zu implementieren.
Dies ging schief.
Bei der Implementation hatte ich gemerkt, dass viele von mir ausgedachten Konzepte nicht funktionierten.
Aus diesem Grund geriet ich am Schluss in Zeitnot und konnte meine Änderungen nicht sauber implementieren und testen.

Bei dieser Arbeit habe ich mich deshalb entschieden, meine Aufgabe in Teilziele aufzuteilen.
Zusätzlich habe ich für jedes Teilziel das ich implementieren wollte einen Test geschrieben.
Nach der Implementierung konnte ich dann den Code einfach testen.
Die kontinuierlichen Tests haben mir nicht nur Sicherheit gegeben, sondern immer auch Teilziele, auf die ich hinarbeiten konnte.

Die Simulation mit Gazebo war ein Ziel, dass erst im Verlauf der Arbeit aufgekommen ist, und nicht in meinem Zeitplan eingeplant war.
Da ich aber meine ursprüngliche Aufgabe in Teilziele geteilt habe, konnte ich ein weniger wichtiges \textit{Feature} (die ROS logging Funktion) streichen und die Anbindung an die Simulation stattdessen implementieren.

In dieser Arbeit habe ich besonders bezüglich Arbeitsplanung und wie ich ein Software Projekt aufbaue viel gelernt.


\section{Ausblick}
EEROS ist nun mit sehr vielen Funktionen ausgestattet, die eine möglichst einfache und flexible Anbindung an ein ROS Netzwerk erlauben.

Die Logging Funktion wurde aber noch nicht implementiert.
Der EEROS Logger schreibt bis jetzt die Ausgaben wahlweise in eine Datei oder auf die Konsole.
Wenn der EEROS Logger so umgelenkt werden könnte, dass er alle Ausgaben als ROS Logger Nachrichten ausgeben würde, könnte auch für solche Nachrichten bestehende ROS Software genutzt werden.

Der ROS Logger hat auch noch zusätzliche Funktionen wie zum Beispiel eine \textit{Throttle} und eine \textit{Filter} Funktion.
Solche Funktionen wären auch für den EEROS Logger vorteilhaft.