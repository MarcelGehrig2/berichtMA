\chapter{Der \textit{gdb}-Debugger}
Es gibt diverse Debugger auf dem Markt.
Diese Arbeit beschränke sich aber auf den  \textit{gdb} (GNU-Debugger), da dieser unter der GNU GPL (General Public License) Lizenz steht und somit eine Open Source Software ist.

In diesem Kapitel wird beschrieben, wie das Demoprogramm aus dem Kapitel \ref{section:demoprogrammSTABS} mit dem \textit{gdb} auf den Zynq geladen werden kann.
Zusätzlich werden auch alle unterstützten Funktionen und Grenzen der \textit{gdb}-OpenOCD-Toolchain diskutiert.


\section{Installation des \textit{gdb}}
% TODO aaaaaaaaa
Der \textit{gdb} ist nur ein Teil von einer

Es gibt diverse Versionen des \textit{gdb} für verschiedene Anwendungen.
Die \textit{gdb}-Toolchains folgen meist einem \textit{A-B-C} Namensmuster.
Dabei gilt:
\begin{itemize}
	\item \textbf{A}: Das Target der Toolchain.
	\begin{itemize}
		\item \textit{''arm''} = ARM 32-bit little-endian.
		\item \textit{''aarch64''} = ARM 64-bit, little-endian.
	\end{itemize}
	\item \textbf{B}: Die Organisation, welche die Toolchain veröffentlicht.
	\begin{itemize}
		\item \textit{''linaro''} = Eine Toolchain von der Engineering-Organisation Linaro.
		\item \textit{''''} = Bei einer generischen Toolchain wird kein Distributor angegeben.
	\end{itemize}
	\item \textbf{C}: Das binäre Applikations-Interface (ABI) der Toolchain
	\begin{itemize}
		\item \textit{''linux-gnu''} = Für Linux-Distributionen.
		\item \textit{''eabi''} = ELF-basiertes \textit{''bare-metal''} Interface.
	\end{itemize}
\end{itemize}

Für den Zynq wird eine \textit{''arm-none-eabi''}-Toolchain benötigt.
ARM stellt unter diesem Download-Link\footnote{https://developer.arm.com/open-source/gnu-toolchain/gnu-rm/downloads} eine komplette, fertig kompillierte Toolchain zur Verfügung.
Darin ist nicht nur der \textit{gdb} enthalten, sondern auch noch Programme wie z.B. das komplette gcc-Compiler-Paket und auch Hilfsprogramme wie \textit{''readelf''} und \textit{''objdump''}.

% TODO gdb herunterladen und installieren



\section{\textit{gdb}-Anwendungsbeispiel mit \textit{''loopWithSTABS''} auf dem Zybo}
\label{section:anwendungsbeispielGdb}
Mit folgenden Schritten kann das kompilierte Programm \textit{''loopWithSTABS''} aus dem Kapitel \ref{section:demoprogrammSTABS} auf den Zynq geladen und debuggt werden:

\begin{enumerate}
	\item Die notwendige Software, wie im Kapitel \ref{kapitel:SoftwareinstallationOpenOCDToolchain} beschrieben, installieren.
	\item Das Zybo per USB-Kabel mit dem PC verbinden.
	\item OpenOCD in der Shell mit dem Befehl \textit{''openocd -f zybo-ftdi.cfg -f zybo.cfg''} starten.
	       Dazu müssen sich die beiden Konfigurationsdateien \textit{''zybo-ftdi.cfg''} und \textit{''zybo.cfg''} (siehe Anhang \ref{anhang:zybo-ftdi.cfg} und Anhang \ref{anhang:zybo.cfg}) im gleichen Ordner wie das \textit{''openocd''}-Binary befinden.
	\item In einer zweiten Shell \textit{gdb} starten.
	       Dazu kann das Shell-Script \textit{''startGdb.ps1''} aus dem Anhang \ref{anhang:startGdb.ps1} genutzt werden.
	       Die Pfade im Script müssen angepasst werden.
	       Die Konfigurationsdatei \textit{''gdbInit.txt''} (siehe Anhang \ref{anhang:gdbInit.txt}) muss im aktiven Ordner vorhanden sein.
	       Alle Pfade in der Konfigurationsdatei müssen ebenfalls angepasst werden.
	\item Im \textit{''gdbInit.txt''} wir die ELF-Datei \textit{''loopWithSTABS''} mit der Instruktion \texttt{''file M:/MA/stabs/loopWithSTABS''} in den \textit{gdb} geladen. Die Instruktion \texttt{''load''} lädt dann das Segment \textit{''.text''} mit dem ausführbaren Code direkt in den Speicher des Zynq.
	\item Die Applikation kann jetzt mit dem \textit{gdb} auf dem Zybo debuggt werden.
\end{enumerate}


\section{Test der \textit{gdb}-Funktionen}
% TODO nur hw breakpoints funktionieren	
In diesem Kapitel werden alle aus dem Kapitel \ref{section:demoprogrammSTABS} geforderten Funktionen getestet.
Als Ausgangspunkt dient das Anwendungsbeispiel aus dem Kapitel \ref{section:anwendungsbeispielGdb}.
\textit{gdb} kann mit dem Befehlt \texttt{''q''} beendet und dann neu gestartet werden, damit die Ausgangslage bei jedem Test identisch ist.

\subsection{Breakpoint}
Das Programm stoppt bei einer gewünschten Zeile im Java-Sourcecode.

\subsection{Source lookup}
Wenn das Programm gestoppt wird, kann die entsprechende Zeile im Java-Sourcecode angezeigt werden.

\subsection{Single-Stepping}
Nur eine Zeile im Java-Sourcecode ausführen und dann pausieren.

\subsection{Variable auslesen}
Eine Java-Variable, z.B. ein Integer, auslesen.

\subsection{Variable manipulieren}
Eine Java-Variable verändern.

\subsection{Prozessor-Register auslesen}
Ein Register der CPU auslesen.






