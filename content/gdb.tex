\chapter{Der \textit{gdb}-Debugger}
Es gibt diverse Debugger auf dem Markt.
Diese Arbeit beschränke sich aber auf den  \textit{gdb} (GNU-Debugger), da dieser unter der GNU GPL (General Public License) Lizenz steht und somit eine Open Source Software ist.

In diesem Kapitel wird beschrieben, wie das Demoprogramm aus dem Kapitel \ref{section:demoprogrammSTABS} mit dem \textit{gdb} auf den Zynq geladen werden kann.
Zusätzlich werden auch alle unterstützten Funktionen und Grenzen der \textit{gdb}-OpenOCD-Toolchain diskutiert.


\section{Installation des \textit{gdb}}
% TODO aaaaaaaaa
Der \textit{gdb} ist nur ein Teil von einer

Es gibt diverse Versionen des \textit{gdb} für verschiedene Anwendungen.
Die \textit{gdb}-Toolchains folgen meist einem \textit{A-B-C} Namensmuster.
Dabei gilt:
\begin{itemize}
	\item \textbf{A}: Das Target der Toolchain.
	\begin{itemize}
		\item \textit{''arm''} = ARM 32-bit little-endian.
		\item \textit{''aarch64''} = ARM 64-bit, little-endian.
	\end{itemize}
	\item \textbf{B}: Die Organisation, welche die Toolchain veröffentlicht.
	\begin{itemize}
		\item \textit{''linaro''} = Eine Toolchain von der Engineering-Organisation Linaro.
		\item \textit{''''} = Bei einer generischen Toolchain wird kein Distributor angegeben.
	\end{itemize}
	\item \textbf{C}: Das binäre Applikations-Interface (ABI) der Toolchain
	\begin{itemize}
		\item \textit{''linux-gnu''} = Für Linux-Distributionen.
		\item \textit{''eabi''} = ELF-basiertes \textit{''bare-metal''} Interface.
	\end{itemize}
\end{itemize}

Für den Zynq wird eine \textit{''arm-none-eabi''}-Toolchain benötigt.
ARM stellt unter diesem Download-Link\footnote{https://developer.arm.com/open-source/gnu-toolchain/gnu-rm/downloads} eine komplette, fertig kompillierte Toolchain zur Verfügung.
Darin ist nicht nur der \textit{gdb} enthalten, sondern auch noch Programme wie z.B. das komplette gcc-Compiler-Paket und auch Hilfsprogramme wie \textit{''readelf''} und \textit{''objdump''}.

% TODO gdb herunterladen und installieren



\section{\textit{gdb}-Anwendungsbeispiel mit \textit{''loopWithSTABS''} auf dem Zybo}
\label{section:anwendungsbeispielGdb}
Mit folgenden Schritten kann das kompilierte Programm \textit{''loopWithSTABS''} aus dem Kapitel \ref{section:demoprogrammSTABS} auf den Zynq geladen und debuggt werden:

\begin{enumerate}
	\item Die notwendige Software, wie im Kapitel \ref{kapitel:SoftwareinstallationOpenOCDToolchain} beschrieben, installieren.
	\item Das Zybo per USB-Kabel mit dem PC verbinden.
	\item OpenOCD in der Shell mit dem Befehl \textit{''openocd -f zybo-ftdi.cfg -f zybo.cfg''} starten.
	       Dazu müssen sich die beiden Konfigurationsdateien \textit{''zybo-ftdi.cfg''} und \textit{''zybo.cfg''} (siehe Anhang \ref{anhang:zybo-ftdi.cfg} und Anhang \ref{anhang:zybo.cfg}) im gleichen Ordner wie das \textit{''openocd''}-Binary befinden.
	\item In einer zweiten Shell \textit{gdb} starten.
	       Dazu kann das Shell-Script \textit{''startGdb.ps1''} aus dem Anhang \ref{anhang:startGdb.ps1} genutzt werden.
	       Die Pfade im Script müssen angepasst werden.
	       Die Konfigurationsdatei \textit{''gdbInit.txt''} (siehe Anhang \ref{anhang:gdbInit.txt}) muss im aktiven Ordner vorhanden sein.
	       Alle Pfade in der Konfigurationsdatei müssen ebenfalls angepasst werden.
	\item Im \textit{''gdbInit.txt''} wir die ELF-Datei \textit{''loopWithSTABS''} mit der Instruktion \texttt{''file M:/MA/stabs/loopWithSTABS''} in den \textit{gdb} geladen. Die Instruktion \texttt{''load''} lädt dann das Segment \textit{''.text''} mit dem ausführbaren Code direkt in den Speicher des Zynq.
	\item Die Applikation kann jetzt mit dem \textit{gdb} auf dem Zybo debuggt werden.
\end{enumerate}


\section{Test der \textit{gdb}-Funktionen}
In diesem Kapitel werden alle aus dem Kapitel \ref{section:demoprogrammSTABS} geforderten Funktionen getestet.
Als Ausgangspunkt dient das Anwendungsbeispiel aus dem Kapitel \ref{section:anwendungsbeispielGdb}.
\textit{gdb} kann mit dem Befehlt \texttt{''q''} beendet und dann neu gestartet werden, damit die Ausgangslage bei jedem Test identisch ist.


Für die bessere Übersicht wird hier nochmals der Java-Code des Demo-Programms \textit{''loop.java''} aufgelistet:
\lstset{language=java}
\begin{lstlisting}
static void reset() {



	US.PUTGPR(SP, stackBase + stackSize - 4);	// set stack pointer
	
	int x00 = 0;
	int x01 = 1;
	int x02 = 2;
	
	x00++;
	x01++;
	x02++;
	
	int x100 = 100;
	for(int i=0; i<10; i++){
		x100 += 10;
   }
		
	x100++;
	x100++;
	x100++;
	x100++;
	x100++;

	US.ASM("b -8"); // stop here
}
\end{lstlisting}


\subsection{Durchführung des \textit{gdb}-Tests}
Mit \texttt{''list''} kann der Sourcecode des Programmes angezeigt werden.
\texttt{''list 10''} zeigt den Sourcecode ab der 10. Zeile an.
Ein Hardware-Breakpoint auf Zeile 11 kann mit \texttt{''hbreak 11''} erstellt werden.
Wird das Programm mit \texttt{''c''} gestartet, dann wird die Ausführung gestoppt, sobald die 11. Zeile des Sourcecodes erreicht wurde.
\textit{gdb} zeigt dann an, dass die nächste Zeile \texttt{''x00++;''} sein wird.

Mit \texttt{''p x00''} wird der Inhalt der Variable \texttt{''x00''} angezeigt.
Führt man mit \texttt{''s''} einen einzelnen Step, also eine Zeile im Sourcecode aus, dann erhöht sich der Wert der Variable \texttt{''x00''} um 1.
Das kann mit \texttt{''p x00''}  wieder überprüft werden.

% TODO bei x01 funktioniert nicht, da in den selben registern

Ein weiterer Hardware-Breakpoint auf Zeile 17 (\texttt{''break 17}) stoppt das Programm innerhalb der For-Loop.
Die Variable \texttt{''i''} zeigt zu diesem Zeitpunkt wie erwartet ''0''.
Wird das Programm fortgesetzt, dann stoppt das Programm wieder auf der Zeile 17 und \texttt{''i''} zeigt ''1''.
Die Variable \texttt{''i''} kann mit \texttt{''set var i=9''} gesetzt werden.
Da mit ''i=9'' die Abbruchbedingung der For-Loop erfüllt ist, wird der Breakpoint nicht mehr erreicht, wenn das Programm weiter ausgeführt wird.
Das Programm hängt jetzt auf der letzten Zeile des Programms fest, und kann mit der Tastenkombination \textit{ĈTRL + C} gestoppt werden.

Das Schlüsselwort \texttt{''monitor''} kann genutzt werden, um OpenOCD aus dem \textit{gdb} heraus direkt ein Befehl zu erteilen.
So kann mit \texttt{''monitor reg''} der OpenOCD-Befehl \texttt{''reg''} genutzt werden, um alle Register anzuzeigen.

\textbf{Hinweis}: 
Seit der \textit{gdb}-Version 8 funktionieren Software-Breakpoints (z.B \texttt{''break 12''}) nicht mehr.
Bei einem Software-Breakpoint wird eine Instruktion mit einer speziellen Instruktion ersetzt, die dann das Programm stoppt und den Debugger triggert.
Das funktioniert bei allen \textit{gdb}-Versionen.
Ab der \textit{gdb}-Version 8 wird diese Instruktion aber nicht mehr mit der alten, gültigen Instruktion ersetzt.
Aus diesem Grund kann dann das Programm nicht mehr weiter ausgeführt werden.
Die Hardware-Breakpoints funktionieren bei allen Versionen.


\subsection{Fazit des \textit{gdb}-Tests}
Alle geforderten Funktionen des Debuggers können grundsätzlich genutzt werden.

Bei \textit{gdb}-Versionen die neuer als Version 8 können aber nur die Hardware-Breakpoints verwendet werden.
Software-Breakpoints könnten aber auch verwendet werden, wenn die ersetzte Instruktion manuell wieder hergestellt wird.



