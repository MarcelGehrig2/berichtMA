\chapter{Einleitung}


\section{Stand der Technik}
% deep
Das Projekt \textit{deep}\footnote{http://www.deepjava.org/start} ist eine Cross Development Plattform, die es erlaubt, ein Java Programm direkt auf einem Prozessor auszuführen.
Es ermöglicht einem Entwickler ein Java Programm zu schreiben, welches direkt auf einem Prozessor läuft und echtzeitfähig ist.
Zur Zeit wird dieses Projekt an der NTB für die Ausbildung von Systemtechnik-Studenten verwendet.
Es erlaubt einfach und schnell Robotersteuerungen und Regelungen zu implementieren, ohne dass sich der Entwickler mit den Eigenarten von C und C++ Programmen auseinandersetzen muss.

% debuging
\textit{deep} unterstützt einige grundlegende Debuging-Funktionen.
Mit einer mehreren tausend Franken teuren Abatronsonde kann der Speicher und die Register des Prozessors ausgelesen und auch geschrieben werden.
Der aktuelle Debugger unterstützt keine Features wie \textit{Breakpoints} oder \textit{Sourcecode-Navigation}, die aus Debuggern wie dem \textit{gdb}\footnote{https://www.gnu.org/software/gdb/} bekannt sind.

% TODO3: JTAG


\section{Motivation}
Aktuell ist \textit{deep} nur mit der PowerPC-Architektur kompatibel.
PowerPC Prozessoren sind aber nicht mehr weit verbreitet und sehr teuer.
Die an der NTB verwendeten PowerPC-Prozessoren sind zwar leistungsstark, aber veraltet und teuer.

Aus diesem Grund wird \textit{deep} zurzeit für die ARM-Architektur erweitert.
Da die ARM-Architektur bei eingebetteten Prozessoren am weitesten verbreitet ist, ist auch die Auswahl an günstiger und leistungsstarker Hardware sehr gross.
Dank der grossen Auswahl von ARM-Prozessoren können sehr günstige oder auch sehr leistungsstarke Prozessoren ausgewählt werden.

% debugging
\textit{deep} ist ein Open-Source-Projekt, welches auch für den Unterricht verwendet wird.
Damit nicht für jeden Studenten teure Debugging-Hardware gekauft werden muss, ist eine kostengünstige Alternative wünschenswert.

% TODO2: ersatz für zielorientiert
Java ist im Gegensatz zu C und C++ eine sehr zielorientierte Sprache.
% TODO1 man ersetzen
Beim Java muss man sich nicht so detailliert um Ressourcen, wie Speicher und Hardwareschnittstellen kümmern, wie in C-orientierten Sprachen.
Dieser Aspekt der Einfachheit soll auch beim Debugger beibehalten werden.
Zusätzlich zum direkten Speicherauslesen sollen auch Variablen gelesen und geschrieben werden können.
Eine native \textit{Sourcecode-Navigation} in Eclipse vereinfacht die Entwicklung einer \textit{deep}-Applikation sehr.



\section{Zielsetzung}
% Hardware
Bei dieser Arbeit werden mehrere Ziele verfolgt, die aufeinander aufbauen.


\begin{enumerate}
\item Passende Hardware (Experimentierboard) finden, welche auch im Unterricht verwendet werden kann.
\item Das grundlegende Debug-Interface, welches bereits für PowerPC existiert, für die ausgewählte Hardware anpassen. Dieses Interface soll für die Entwicklung von \textit{deep} möglichst bald einsatzbereit sein.
\item Den GNU-Debugger (\textit{gdb}) mit einem Programm verwenden, das vom \textit{deep}-Compiler übersetzt wurde. Dazu soll vorerst das Command-Line-Interface (CLI) des \textit{gdb} genutzt werden.
\item Den \textit{gdb} in das Eclipse-Plugin von \textit{deep} integrieren, damit der Debugger direkt aus Eclipse verwendet werden kann.
\end{enumerate}


% TYPISCHE BEDÜRFNISSE ALS LEHRMITTEL ROBOTIK






% Aus diesem Grund wird \textit{deep} um einen Codegenerator für ARM-Prozessoren erweitert.
% In dieser Masterthesis soll zuerst eine passende Hardware gefunden werden, auf der die ARM-Implementierung entwickelt und getestet werden kann.
% Die Hardware sollte optimal für den Unterricht und das Systemtechnikprojekt passen.
% Zudem muss sie auch für die Steuerung von komplexen echtzeitfähigen Steuer- und Regelungstechnikanwendungen verwendet werden können.
% Für den ausgewählten ARM-Prozessor muss natürlich auch die Runtime-Umgebung angepasst werden.
% Ein Board Support Package ermöglicht den Zugriff auf die Peripherie des Prozessors.
% Zwingend erforderlich ist die Integration unserer «flink» FPGA-Anbindung über ein geeignetes Hardwareinterface.
% Das Board Support Package muss die wichtigsten Grundfunktionen für einen Betrieb anbieten.
% Debugging ist sowohl für die Weiterentwicklung „deep“ wie auch für die Entwicklung von Applikationen sehr wichtig.
% Debugging soll direkt aus Eclipse möglich sein.
% Dazu sollen das On-Chip Debugging Framework (OCD) und der GNU-Debugger (gdb) eingesetzt werden.
