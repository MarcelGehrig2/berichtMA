\chapter{Einleitung}


\section{Stand der Technik}
% deep
Das Projekt \textit{deep} \cite{bib:DeepHome} ist eine Cross Development Plattform, die es erlaubt, ein Java Programm direkt auf einem Prozessor auszuführen.
Das \textit{deep} Projekt ermöglicht es einem Entwickler eine Java Programm zu schreiben, welches direkt auf einem Prozessor läuft und Echtzeit-Fähigkeiten hat.
Zur Zeit wird dieses Projekt in der NTB für die Ausbildung Von Systemtechnik Studenten verwendet.
Es erlaubt einfach und schnell Robotersteuerungen und Regelungen zu implementieren, ohne dass man sich mit den Eigenarten von C und C++ Programmen auseinandersetzen muss.

% debuging
\textit{deep} unterstützt einige grundlegende Debuging-Funktionalitäten.
Mit einer mehreren tausend Franken teuren Abatronsonde kann der Speicher und die Register des Prozessors ausgelesen und auch geschrieben werden.
Auch werden keine \textit{Breakpoints} oder \textit{Source Code Navigation} unterstützt, so wie man es aus bekannten Debuggern wie dem \textit{gdb}\footnote{https://www.gnu.org/software/gdb/} kennt.

% JTAG


\section{Motivation}
% ARM
Aktuell ist \textit{deep} nur mit der PowerPC Architektur kompatibel.
PowerPC Prozessoren sind aber nicht mehr weit verbreitet und sehr teuer.
Die an der NTB verwendeten PowerPC-Prozessoren sind zwar leistungsstark, aber teuer und veraltet.

Aus diesem Grund wird \textit{deep} für ARM-Prozessoren erweitert.
Da die ARM-Architektur bei eingebetteten Prozessoren am weitesten verbreitet ist, ist auch die Auswahl an günstiger und leistungsstarker Hardware sehr gross.
Mit grosse Flexibilität bei der Auswahl von ARM-Prozessoren kann wahlweise sehr günstige oder auch sehr leistungsstarke Prozessoren verwendet werden.

% debugging
\textit{deep} ist ein Open-Source-Projekt welches auch für den Unterricht verwendet wird.
Damit nicht für jeden Student teure Debugging-Hardware gekauft werden muss, ist eine kostengünstige Alternative wünschenswert.

Java ist im Gegensatz zu C und C++ eine sehr zielorientierte Sprache.
Bei Java muss man sich nicht so detailliert um Ressourcen wie Speicher und Hardwareschnittstellen kümmern wie in C-orientierten Sprachen.
Dieser Aspekt soll auch beim Debugger beibehalten werden.
Zusätzlich zum direkten Speicher Auslesen sollen auch Variablen gelesen und geschrieben werden können.
Eine native \textit{Source Code Navigation} in Eclipse vereinfacht die Entwicklung einer deep-Applikation sehr.



\section{Zielsetzung}
% Hardware
Bei dieser Arbeit werden mehrere Ziele verfolgt, die aufeinander aufbauen.


\begin{enumerate}
\item Passende Hardware (Experimentierboard) finden, welche auch im Unterricht verwendet werden kann.
\item Grundlegendes Debug-Interfeace, welches bereits für PowerPC existiert, für die ausgewählte Hardware anpassen.
\item Den GNU-Debugger (gdb) mit einem Programm verwenden, dass vom deep-Compiler übersetzt wurde. Dazu soll vorerst das Command-Line-Interface (CLI) des gdb genutzt werden.
\item Den gdb in das Eclipse Plug-In von deep integrieren, damit der Debugger direkt aus Eclipse verwendet werden kann.
\end{enumerate}


% debuging






% Aus diesem Grund wird \textit{deep} um einen Codegenerator für ARM-Prozessoren erweitert.
% In dieser Masterthesis soll zuerst eine passende Hardware gefunden werden, auf der die ARM-Implementierung entwickelt und getestet werden kann.
% Die Hardware sollte optimal für den Unterricht und das Systemtechnikprojekt passen.
% Zudem muss sie auch für die Steuerung von komplexen echtzeitfähigen Steuer- und Regelungstechnikanwendungen verwendet werden können.
% Für den ausgewählten ARM-Prozessor muss natürlich auch die Runtime-Umgebung angepasst werden.
% Ein Board Support Package ermöglicht den Zugriff auf die Peripherie des Prozessors.
% Zwingend erforderlich ist die Integration unserer «flink» FPGA-Anbindung über ein geeignetes Hardwareinterface.
% Das Board Support Package muss die wichtigsten Grundfunktionen für einen Betrieb anbieten.
% Debugging ist sowohl für die Weiterentwicklung „deep“ wie auch für die Entwicklung von Applikationen sehr wichtig.
% Debugging soll direkt aus «eclipse» möglich sein.
% Dazu sollen das On-Chip Debugging Framework (OCD) und der GNU-Debugger (gdb) eingesetzt werden.
