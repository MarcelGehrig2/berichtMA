\chapter{OpenOCD}
\section{Einleitung}
% OpenOCD ist ein ''On-Chip Debugger''.\cite{bib:OpenOCDHome}
% Diese Bezeichnung ist allerdings etwas irreführend.
OpenOCD\cite{bib:OpenOCDHome} bildet den Software Teil von einem Debugger.
Zusammen mit einem Hardware Adapter bildet OpenOCD einen vollständigen Debugger und kann als Ersatz für einen teuren Debugger wie beispielsweise dem BDI 3000 von Abatron verwendet werden.

Der Adapter bildet dabei das elektrische Interface zum Prozessor und muss auch auf den Prozessor abgestimmt sein.
Relevant sind dabei unter anderem der Transport Layer (JTAG/SWD) das elektrische Potential und natürlich auch der Physikalischer Stecker.
In den meisten Fällen basieren solche Adapter, wenn sie zusammen mit OpenOCD verwendet werden, auf dem FT2232 Chip von FTDI.
Solch ein generischer Adapter ist in der Abbildung \ref{fig:GenerischerFT2232Adapter} zu sehen.

\begin{figure}[htbp]
	\centering
		% \includegraphics[width=\textwidth,height=\textheight,keepaspectratio]{images/JTAGAdapter.jpg}
		\includegraphics[width=7cm,keepaspectratio]{images/JTAGAdapter.jpg}
	\caption{Generischer JTAG Adapter mit einem FTDI FT2232\cite{bib:ebayJTAGAdapter}}
	\label{fig:GenerischerFT2232Adapter}
\end{figure}

Bei Experimentierboards ist der FT2232 oft auch direkt auf das Board aufgelötet.
So kann eine einfache USB Verbindung genutzt werden, um den Prozessor zu debuggen.
Beim Zybo wurde ebenfalls dieser Ansatz verfolgt.
Aus diesem Grund reicht ein einfaches USB Kabel um den Prozessor auf dem Zybo auf einer Hardware-Ebene debuggen zu können.

\section{Installation}
\subsection{Installation - OpenOCD}
OpenOCD kann direkt vom Sourcecode kompiliert werden\footnote{http://sourceforge.net/p/openocd/code/} oder auch als vorkompiliertes Binary herunter geladen werden.
Für diese Arbeit wurde das vorkompilierte Windows Binary für ARM Cores Version 0.10.0 von folgender URL verwendet:\\
\textit{http://www.freddiechopin.info/en/download/category/4-openocd?download=154\%3Aopenocd-0.10.0}

Das eigentliche Binary befindet sich im Ordner:\\
\texttt{/openocd-0.10.0/bin-x64/} 

Das User Manual befindet sich im Ordner:\\
\texttt{/openocd-0.10.0/} 


\subsection{Installation - USB Driver WinUSB}
Damit OpenOCD mit dem FT2232 Chip kommunizieren kann, werden noch die richtigen USB Treiber benötigt.
Die Installation der Treiber ist am einfachsten mit den \textit{USB Driver Tool}, welches man unter folgender Adresse findet:\\
\textit{http://visualgdb.com/UsbDriverTool/}

Das Zybo muss per USB mit dem PC verbunden sein, damit der Treiber installiert werden kann.
Wenn der Jumper '\textit{J15}' auf USB gesetzt ist, dann wird keine zusätzliche Stromversorgung für das Zybo benötigt.

Öffnet man das \textit{USB Driver Tool} werden alle USB Devices aufgelistet.
Das Device mit der \textit{Vendor ID=0403, Device ID=6010} und \textit{Interface 0} ist das JTAG Interface vom FT2232.
Mit einem Rechtsklick darauf kann man den \textit{Install WinUSB} Treiber auswählen und installieren.
Abbildung \ref{fig:InstallWinUSBDriver} zeigt die Liste mit allen USB Devices und das Kontextmenü für die Installation des richtigen Treibers.
Nachdem das Zybo einmal aus- und wieder einschaltet wird, ist der Treiber einsatzbereit.

Das Device mit der \textit{Vendor ID=0403, Device ID=6010} und \textit{Interface \textbf{1}} ist die UART Verbindung zum Prozessor.
Dieser Treiber darf \textbf{nicht} ersetzt werden.

\begin{figure}[htbp]
	\centering
		% \includegraphics[width=\textwidth,height=\textheight,keepaspectratio]{images/JTAGAdapter.jpg}
		\includegraphics[width=12cm,keepaspectratio]{images/InstallWinUSBDriver.png}
	\caption{Installation des \textit{WinUSB} Treibers mit dem \textit{USB Driver Tool}}
	\label{fig:InstallWinUSBDriver}
\end{figure}


\section{OpenOCD Konfiguration}
% TODO: Files anhängen
% nicht einfach
% spezielle sprache jim-tcl
OpenOCD unterstützt eine Vielzahl von Adaptern und Targets (Prozessoren).
Beim Start muss die Software für die verwendete Hardware konfiguriert werden.
Die Konfiguration erfolgt mit Konfigurationsscripts (*.cfg) in der Scriptsprache \textit{Jim-Tcl}.
\textit{Jim-Tcl} ist eine abgespeckte Version von \textit{Tcl}\footnote{http://www.tcl.tk}.

Normalerweise werden die Scripts in die drei Gruppen \textit{interface, board} und \textit{target} aufgeteilt.
So kann einfach ein Script ausgewechselt werden, wenn man den gleichen Adapter aber einen anderen Prozessor verwenden will.
Im Pfad \textit{openocd-0.10.0/scripts} befinden sich eine Sammlung von Konfigurationsscripts für Standardhardware.

Mit folgendem Befehl kann OpenOCD mit Konfigurationsscripten nach Wahl gestartet werden:\\
\texttt{openocd -f script1.cfg -f script2.cfg -f script3.cfg}


\section{OpenOCD Konfiguration - Interface}
Die Interface Konfiguration beschreibt hauptsächlich den verwendeten Adapter.
Da beim Zybo kein Adapter verwendet wird, sondern der aufgelötete FT2232, wird mit diesem Script der FTDI Chip und dessen Anbindung an den Zynq konfiguriert.

Da ein FTDI-Chip als Interface verwendet wird, sollte ein passender Script unter \textit{openocd-0.10.0/scripts/interface/ftdi/} zu finden sein.
Keiner der Scripts passt von Namen her auf \textit{Zynq, Zybo} oder \textit{FT2232}.
Eine Google Suche nach einem passenden Script war erfolgreicher.
Ein Github User mit dem Namen \textit{emard} hat folgenden Script in einem von seinen Repositories gespeichert:

% TODO neutrale text-farben
\textit{ftdi-zybo.ocd:}
\begin{lstlisting}
#
# ZYBO ft2232hq usbserial jtag
#

interface ftdi
ftdi_device_desc "Digilent Adept USB Device"
ftdi_vid_pid 0x0403 0x6010

ftdi_layout_init 0x3088 0x1f8b
#ftdi_layout_signal nTRST -data 0x1000 -oe 0x1000
# 0x2000 is reset
ftdi_layout_signal nSRST -data 0x3000 -oe 0x1000
# green MIO7 LED
ftdi_layout_signal LED -data 0x0010
#ftdi_layout_signal LED -data 0x1000

reset_config srst_pulls_trst
}
\end{lstlisting}




\section{OpenOCD Konfiguration - Board}
\section{OpenOCD Konfiguration - Target}
