\chapter{Einleitung}

\section{Unterschiede zwischen ARM Cortex -A -R und -M}
\subsubsection{Cortex-A}
Sehr gut geeignet für die Verwendung mit einem vollen Betriebssystem wie Windows, Linux oder Android.
Die leistungstärksten ARM Cortex Prozessoren.


\subsubsection{Cortex-R}
Cortex-R werden entwickelt für Echtzeitanwendungen und Sicherheitskritische Applikationen wie Festplattenkontroller und medizinische Geräte.
Sie sind normalerweise nicht mit einer MMU ausgerüstet.
% und können deshalb nich mit einem Linux oder Windows verwendet werden.
Mit einer Taktrate von über 1GHz und einem sehr schnellen Interruptverhalten eignen sich Prozessoren mit einem Cortex-R sehr gut um auf externe Stimuli schnell zu reagieren.

\subsubsection{Cortex-M}
Cortex-M sind mit einer Taktrate um 200Mhz relativ langsam.
Sehr stromsparend und durch die kurze Pipeline haben sie eine deterministische und kurze Interrupt Verzögerung.
Die Prozessoren aus der Cortex-M Reihe unterstützen nur die Thumb Instruktionen und nicht die standard Arm Instruktionen.


\subsubsection{ARM Prozessoren ausserhalb der Cortex Reihe}
%TODO jahreszahl 2004?
Seit XXXX werden alle Kerne in eine der Cortex Gruppen eingeteilt.
Ältere Kerne haben namen wie z.b. ARM7 oder ARM1156T2F-S.
Da solche Designs aus einer Zeit vor XXXX stammen, gilt das Design als veraltet und wird bei dieser Arbeit nicht berücksichtigt.	%TODO jahreszahl


\subsubsection{Übersicht über die}

\begin{table}[]
\centering
\caption{My caption}
\label{my-label}
\begin{tabular}{|l|l|l|}
\hline
  & \textbf{Vorteile}                                                                                                                                                                                                                                    & \textbf{Nachteile}                                                                                                                                                                             \\ \hline
A & \begin{tabular}[c]{@{}l@{}}* Sehr Leistungsstartk\\ * Support für vollwertige Betriebssysteme\\ * Grosse Variation erhältich (Energiesparend /\\  sehr Leistungsstark)\\ * Reichhaltiger Funktionsumfang\\ * NEON und FPU Unterstützung\end{tabular} & \begin{tabular}[c]{@{}l@{}}* Langsamer Context-Switch\\ * Relativ hoher Stromverbrauch\\ * Relativ teuer\\ * Mit GPU erhältlich\\ * Keine DSP Unterstützung\\ * Keine HW-Division\end{tabular} \\ \hline
R & \begin{tabular}[c]{@{}l@{}}* Sehr gut geignet für Echtzeitanwendungen\\ * Sehr schneller Context-Switch\\ * DSP Unterstützung\end{tabular}                                                                                                           & \begin{tabular}[c]{@{}l@{}}* Kleiner Funktionsumfang\\ * Nicht so leistungstark wie Cortex A\\ * Keine Linux Unterstützung\end{tabular}                                                        \\ \hline
M & \begin{tabular}[c]{@{}l@{}}* Sehr schneller Context-Switch\\ * Sehr energiesparend\\ * DSP Unterstützung\end{tabular}                                                                                                                                & \begin{tabular}[c]{@{}l@{}}* Geringe Rechenleistung\\ * Keine Linux Unterstützung\\ * Unterstützt nur Thumb-Instruktionen\end{tabular}                                                         \\ \hline
\end{tabular}
\end{table}




%\section{Cortex-M}
%\textbf{Cortex-M0}
%A very small processor (starting from 12K gates) for low cost, ultra low power microcontrollers and deeply embedded applications
%
%\textbf{Cortex-M0+}
%The most energy-efficient processor for small embedded system. Similar size and programmer’s model to the Cortex-M0 processor, but with additional features like single cycle I/O interface and vector table relocations
%
%\textbf{Cortex-M1}
%A small processor design optimized for FPGA designs and provides Tightly Coupled Memory (TCM) implementation using memory blocks on the FPGAs. Same instruction set as the Cortex-M0
%
%\textbf{Cortex-M3}
%A small but powerful embedded processor for low-power microcontrollers that has a rich instruction set to enable it to handle complex tasks quicker. It has a hardware divider and Multiply-Accumulate (MAC) instructions. In addition, it also has comprehensive debug and trace features to enable software developers to develop their applications quicker
%
%\textbf{Cortex-M4}
%It provides all the features on the Cortex-M3, with additional instructions target at Digital Signal Processing (DSP) tasks, such as Single Instruction Multiple Data (SIMD) and faster single cycle MAC operations. In addition, it also have an optional single precision floating point unit that support IEEE 754 floating point standard
%
%\textbf{Cortex-M7}
%High-performance processor for high-end microcontrollers and processing intensive applications. It has all the ISA features available in Cortex-M4, with additional support for double-precision floating point, as well as additional memory features like cache and Tightly Coupled Memory (TCM)



\subsubsection{STM23}
STM